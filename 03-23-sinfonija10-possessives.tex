\input{\string~/Cloud/LaTeX/Templates/abstract-preamble.tex}

\begin{document}

\noindent\textsf{\textbf{Hungarian possessive constructions, object agreement,
        and specificity}}\\[-.33\baselineskip]

\noindent This paper reports results of a series of surveys of possessive
constructions in Hungarian and their behaviour with respect to object
agreement.  Hungarian has differential object agreement: while all subjects
trigger agreement with the verb (glossed as \Sbj{}), a proper subset of direct
objects (DOs) triggers additional object agreement (glossed as \Obj{}). In
(\nextx a), the verb agrees with the subject only, in (\nextx b), it agrees
with both the subject and the object.

\noindent\begin{minipage}[t]{.5\textwidth}
\pex
    \a
    \begingl
        \gla    Lát-ok egy kutyá-t.//
        \glb    see-\Fsg.\Sbj{} a dog-\Acc{}//
        \glft   `I see a dog.'//
    \endgl
\xe
\end{minipage}
\begin{minipage}[t]{.5\textwidth}
\pex[exno=, exnoformat=X]
    \a[label=b]
    \begingl
        \gla    Lát-om a kutyá-t.//
        \glb    see-\Fsg.\Obj{} a dog-\Acc{}//
        \glft   `I see the dog.'//
    \endgl
\xe
\end{minipage}

\noindent Definiteness (as in (\lastx)) is not a perfect predictor of object
agreement. With respect to possessed direct objects,
\textcite{Szabolcsi1994,EKiss2000} suggest that there are two varieties of
Hungarian: one group, the \enquote{standard}, requires object agreement with
\emph{all possessed DOs}, independently of their interpretation. Another
variety, the \enquote{non-standard}, makes a semantic distinction: non-specific
possessed DOs co-occur with subject agreement only (\Sbj), while specific and
definite possessed DOs trigger object agreement (\Obj). A relevant example is
shown in (\nextx), which is only acceptable for a subset of Hungarian speakers.

\ex
    \begingl\rightcomment{\Sbj{}, non-specific DO}
        \gla    \ljudge\%Chomsky-nak nem olvas-t-ál vers-é-t.//
        \glb    Chomsky-\Dat{} \Neg{} read-\Pst-\Ssg{} poem-\Poss.\Tsg-\Acc{}//
        \glft   `You have not read a poem of Chomsky's.'\trailingcitation{\parencite[227]{Szabolcsi1994}}//
    \endgl
\xe
In previous surveys, we have not found a significant number of speakers who
accept examples like (\lastx). One reason for this is the absence of a
determiner (participants consistently judged examples with determiners higher
than those without). Another reason is the ambiguity of examples like (\lastx):
for \enquote{standard} speakers, there is no morphosyntactic difference between
specific and non-specific readings, which makes it difficult to control for
effects of interpretation.

\section{Data} To avoid this problem, we conducted further tests with target
sentences including imperatives and possessed DOs, as shown in (\nextx).
Indefinite objects in such contexts are obligatorily interpreted as
non-specific, e.g.\ \emph{Bring me a doctor!} \parencite[154]{Abbott2010}.

\pex
    \a
    \begingl\rightcomment{\Sbj{}, non-specific DO}
        \gla    Hív-j-ál be egy titkárnő-m-et.//
        \glb    call-\Imp-\Ssg.\Sbj{} in a secretary-\Poss.\Fsg-\Acc{}//
        \glft   `Call in a secretary of mine.'//
    \endgl
    \a
    \begingl\rightcomment{\Obj{}, non-specific DO}
        \gla    Hív-d be egy titkárnő-m-et.//
        \glb    call-\Imp.\Ssg.\Obj{} in a secretary-\Poss.\Fsg-\Acc{}//
        \glft   `Call in a secretary of mine.'//
    \endgl
\xe
This makes it possible to \textbf{control for specificity} and makes a number
of predictions about expected judgments. First, non-standard speakers should
judge (\lastx a) as significantly better than (\lastx b). Second, if standard
speakers treat all possessed DOs alike independently of interpretation, they
should judge (\lastx b) as significantly better than (\lastx a). Third, if
standard speakers are sensitive to the interpretation of possessed DOs, they
should judge neither type as acceptable: (\lastx a) is ruled out because
standard speakers do not allow for \Sbj{} with possessed DOs, and (\lastx b) is
ruled out if standard speakers require possessed DOs to be specific.

\section{Tests} Participants were asked to judge sentences like (\lastx) on a
seven-point Likert scale, two choose one of two alternatives like (\lastx a,b),
and to fill in a verb form in a gap: \underline{\hphantom{2em}} \emph{possessed
    DO}.

\section{Results}

\section{Conclusions} Our results suggest that \ldots{}

\newrefcontext[sorting=nyt]
\printbibliography

\end{document}
